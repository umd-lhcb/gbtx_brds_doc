\subsubsection{Shorts versus GND}
We shouldn't connect any power to the VLDB when doing this test.
Check to ensure no shorts between ground and following pins/points:

\begin{enumerate}
    \item \texttt{D1} (1.5V)
    \item \texttt{D2} (3.3V)
    \item \texttt{C3} (1.5V---\texttt{GBTx\_D} and \texttt{SCA\_D})
    \item \texttt{C59} (1.5V---\texttt{SCA\_A})
    \item \texttt{C41} (1.5V---\texttt{GBTx\_C})
    \item \texttt{TP11} (2.5V---\texttt{VTRx})
    \item \texttt{C44} (2.5V---\texttt{VTRx\_Tx})
    \item \texttt{C45} (2.5V---\texttt{VTRx\_Rx})
\end{enumerate}

\begin{leftbar}
    Most capacitors are on back of board. Capacitors will have two sides with
    different resistance amounts, fine as long as one isn't 0.
\end{leftbar}

\begin{leftbar}
    All tests done at 5V. VLDB has min of 5V and max of 12V with current limit
    of 3A
\end{leftbar}

Tested in this section:
\begin{itemize}
    \item Power ports shortcuts
    \item GBTx, SCA and VTRx main power network shortcuts
\end{itemize}


\subsubsection{DCDC-Power check}
Connect the 5V power supply to the 2 DCDCs (1V5 and 2V5/3V5), then check voltage
level and LEDs at following points:

\begin{enumerate}
    \item LED \texttt{LD5} ON (1.5V)
    \item LED \texttt{LD6} ON (2.5V/3.5V)
    \item \texttt{D1} (5V)
    \item \texttt{TP11} (2.5V/3.5V)
    \item \texttt{TP9} (1.5V)
    \item \texttt{TP12} (2.5V---OK PGOOD)
    \item \texttt{TP10} (1.5V---OK PGOOD)
\end{enumerate}

Tested in this section:
\begin{enumerate}
    \item Voltage level on power ports
    \item Voltage level on DCDC outputs
\end{enumerate}

\subsubsection{DCDC-Enable pins}
To make sure the DCDC modules actually deliver power to the VLDB, do the
following steps:

Install jumpers at \texttt{ST2} (1.5V EN) and \texttt{ST3} (2.5V EN).

Connect USB-\itwoc\ dongle.
\begin{leftbar}
    You should see:
\begin{enumerate}
    \item LED \texttt{LD5} ON (1.5V)
    \item LED \texttt{LD6} ON (2.5V)
\end{enumerate}
\end{leftbar}

Send 1.5V reset with \textbf{GBTx Programmer}.
\begin{leftbar}
    You should see:
\begin{enumerate}
    \item LED \texttt{LD5} OFF for 0.5 seconds, then ON (1.5V)
\end{enumerate}
\end{leftbar}

Send 2.5V reset with the same software.
\begin{leftbar}
    You should see:
\begin{enumerate}
    \item LED \texttt{LD6} OFF for 0.5 seconds, then ON (2.5V)
\end{enumerate}
\end{leftbar}

Tested in this section:
\begin{enumerate}
    \item Power delivery to VLDB
    \item Ability to control DCDC modules with GBTx via \itwoc.
    \item \itwoc\ connectivity
\end{enumerate}


\subsubsection{Configure GBTx \itwoc\ address}
To do so, make sure to set \texttt{StateOverride} and \texttt{RefClkSelect} in
\texttt{SW2} to 0.

The GBTx \itwoc\ address is controlled by \texttt{SW3[5:8]}. A few examples:

\begin{itemize}
    \item Set \texttt{SW3} to 0001, scan GBTx, \itwoc\ address is 1.
    \item Set \texttt{SW3} to 1110, scan GBTx, \itwoc\ address is 14.
\end{itemize}

\begin{leftbar}
    If we flip the DIPs to ON, it means that we are pulling them low.
\end{leftbar}

\begin{leftbar}
    \texttt{ST3} is mounted which means that the GBTx chip is held under reset
    until the PGOOD (voltage is good) is issued.
\end{leftbar}

Tested in this section:
\begin{enumerate}
    \item Functionality of switch \texttt{SW3}
\end{enumerate}


\subsubsection{Bring external clock to VLDB}
Flip the \texttt{RefClkSelect} to 1, then connect a external 40.078 MHz clock to
the SMA connectors on the VLDB.
