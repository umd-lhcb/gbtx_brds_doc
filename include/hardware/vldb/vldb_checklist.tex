\subsubsection{Shorts versus GND}

No power connected, no DCDCs connected

Check to ensure no shorts between ground and following pins/points (i.e. Resistance=0)
Note the following: Most capacitors are on back of board. Capacitors will have two sides with different resistance amounts, fine as long as one isn’t 0.

Note: All tests done at 5V. VLDB has min of 5V and max of 12V with current limit of 3A

\begin{enumerate}
    \item D1 (1.5V)
    \item D2 (3.3V)
    \item C3 (1.5V – GBTx_D and SCA_D)
    \item C59 (1.5V – SCA_A)
    \item C41 (1.5V – GBTx_C)
    \item TP11 (2.5V – VTRx)
    \item C44 (2.5V – VTRx_Tx)
    \item C45 (2.5V – VTRx_Rx)
\end{enumerate}

Tested:
\begin{enumerate}
    \item Power ports shortcuts
    \item GBTx, SCA and VTRx main power network shortcuts
\end{enumerate}


\subsubsection{DCDC-Power check}

Connect the 5V power supply and the 2 DCDCs (1V5 and 2V5/3V5)

Check voltage level and LEDs at following points:

\begin{enumerate}
    \item LED LD5 ON (1.5V)
    \item LED LD6 ON (2.5V/3.5V)
    \item D1 (5V)
    \item TP11 (2.5V/3.5V)
    \item TP9 (1.5V)
    \item TP12 (2.5V – OK PGOOD)
    \item TP10 (1.5V – OK PGOOD)
\end{enumerate}

Tested:
\begin{enumerate}
    \item Voltage level on power ports
    \item Voltage level on DCDC outputs
\end{enumerate}

\subsubsection{DCDC-Enable pins}

Install jumpers at ST2 (1.5V EN) and ST3 (2.5V EN)

Connect USB-I2C dongle

\begin{enumerate}
    \item LED LD5 ON (1.5V)
    \item LED LD6 ON (2.5V)
\end{enumerate}

Send 1.5V reset
\begin{enumerate}
    \item LED LD5 OFF for 0.5 seconds, then ON (1.5V)
\end{enumerate}

Send 2.5V reset
\begin{enumerate}
    \item LED LD6 OFF for 0.5 seconds, then ON (2.5V)
\end{enumerate}

Tested:
\begin{enumerate}
    \item Power-down-from-dongle pins
    \item Power-up-from-dongle pings
\end{enumerate}


\subsubsection{GBTx I2C test}

Note that for this section, 0 = left and 1 = right
Set StateOverride and RefClkSelect in SW2 to 0

Set I2C address (SW3) to 0001
Read GBTx Status, address I2C = 1, status = 5

Set I2C address to 1110
Read GBTx status, adress I2C = dx14, status = 5

(ST3 is mounted which means that the GBTx chip is reset until the PGOOD is issued)

Tested:
\begin{enumerate}
    \item Switch SW3 address
    \item I2C SDA and SCL pins
\end{enumerate}

\subsubsection{Resets, status LEDs, REFclk SMAs and config pins}

Set the pin CONFIGSELECT to 1 (SW2). Set TxDATAVALID to 1 (SW2)
MODES in SW1 = 0010

If there is a DCDC 2.5V, connect the VTRx (RadHard)
If there is a DCDC 3.3V, connect an SFP

Configure the GBTx in (Tranceiver) and LOOPBACK A (maybe a reset is necessary). This implies that the mode switches (SW1) must be 0010.

Configure generator 1 with 120 MHz:

FPGA Configuration:
\being{enumerate}
    \item Connect the 123 MHz to the Refclkin of the FPGA tranceiver
    \item Loopback the SFP on the KC705 board and check on chipscope that the recovered signal is ok
    \item Connect the GBT-FPGA KC705 to the GBTx (VLDB/SFP to the KC705/SFP, both directions), with a DATA header (0101)
\end{enumerate}

Check: LD1, LD2, LD3 ON

Change the header type on the GBT-FPGA to IDLE

Check: LD2 = OFF

Press PB1 during a moment, and write register 365 with 0xAA value (configDOne flag up)

Check: LD1, LD2, LD3 ON

REFCLKSELECT => 1

Connect REFCLK SMAs (PX1 and PX2) with a 40.078 MHz generator (CG635)

Reset PB1, write register 365 with 0xAA value (configDone flag up)

Check: LD1, LD2, LD3 ON

\subsubsetion{Bullseye}
Keep the same configuration (trscv) + testoutput = TxRdy (reg 280 = 30) control and all ClkDES and the Testclkout as 40 MHz

\being{enumerate}
    \item Testclockout (40 MHz, SE)
    \item Testoutput should be 1 (then reset => 0) and then back 1
    \item ClkDes0
    \item ClkDes1
    \item ClkDes2
    \item ClkDes3
    \item ClkDes4
    \item ClkDes5
    \item ClkDes6
    \item ClkDes7
\end{enumerate}

Tested:
\being{enumerate}
    \item Bullseye pins (TP2)
\end{enumerate}

\subsubsection{E-LINKS}
